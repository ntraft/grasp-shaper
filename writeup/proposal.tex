\documentclass[12pt, oneside]{article}
\usepackage[top=1.5cm, bottom=2cm, left=2cm, right=2cm]{geometry}
\geometry{letterpaper}

\title{CPSC 530P Project Proposal}
\author{
  Jolande Fooken
  \and
  Neil Traft
}

\begin{document}
\maketitle

\section*{Introduction}

\section*{Methodology}
We will construct an experimental setup in which an object is placed on a table, and the BarrettHand is manually moved into a position suitable for grasping the object. We will then initiate an automated grasping routine, which we will design, that will enclose the fingers on the object and attempt to lift the object straight upward from the table. This will be done in repeated trials with objects of varying mass and shape.

During the grasp and lift, the routine will automatically modulate the torque applied in each finger joint to deliver the necessary amount of force at each fingertip, such that the object does not slip. These forces will be determined by feedback received from the tactile and torque sensors embedded in each finger of the BarrettHand. Our algorithm will be based on what is currently known about haptic feedback in humans, but will need to be adapted to utilize the more limited forms of sensing found in robots.

If we are successful in constructing an algorithm that modulates grip strength based on tactile feedback, we will extend the system to adapt the grip strategy (the finger and hand pose) based on tactile probing of an object. In this second stage of the project, the hand will still be manually positioned near the object before the start of the procedure, as the focus of our algorithm will be to adapt the \emph{shape} of the grip, not the \emph{approach}.

\section*{Expected Results}
Our goal is to design algorithms for a robotic hand which successfully use tactile information to enhance stability when grasping diverse objects.

In the first stage of the project, we expect to be able to design a feedback controller that modulates grip force based on an estimate of object mass and shape. These estimates will be built from data received from the various tactile sensors mentioned above.

In the second stage of the project, we will design a more sophisticated controller based on the same principles. In addition to adjusting force applied to each finger joint, this second controller will be able to adjust finger and possibly hand orientation toward increased grip stability.

\section*{References}


\end{document}